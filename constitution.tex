\documentclass{article}
\usepackage[utf8]{inputenc}
\usepackage{xcolor}
\usepackage{todonotes}

\title{Constitution of SCS Dean's PhD Student Advisory Committee}
\author{}
\date{Last amended October 2020}

\begin{document}
\setlength{\parskip}{1em}
\maketitle

% purpose 
% id problems relevant to scs phd students
% create Working Groups to tackle those problems
% communicate up to admin and solicit feedback from students
% repeat forever
% groups to solve problems -- self initiated
% random student to spin up new groups
% meet with martial multiple times semester
% there, every student that has something to say + reps from meta Working Groups
% each Working Group determines their own person
% Working Groups dissolve when finished or when everyone leaves
% no standing Committees that are not necessary for the function of the Committee
% need to be completely transparent with non-confidential info, non-confidential is default and most common
% info revolves around website

% TODO:



% - Aug 3 -- have a full draft, present at General Body Meeting, have list of steps we need to take to make us constitution compliant

% - Soon after, have town hall to talk about this with interested students

% - Soon after that, What do we need to tell students during orientation

% - Interface for students to talk to us

\section{Name} \label{sec:name}
This organization is the SCS Dean's PhD student Advisory Committee (DPAC), referred to as the Committee in the remainder of this document. 

\section{Mission Statement} \label{sec:mission}

We are a group of SCS PhD students who advocate for the needs and challenges of SCS PhD students. We ensure that PhD students are heard by the SCS administration, and we work with the administration to propose initiatives to address these challenges and hold the administration accountable for implementation. Our primary goals and responsibilities are to:

1) Strengthen communication between the SCS administration and its PhD students. 
This involves:
\begin{itemize}
    \item Actively soliciting student perspectives
    \item Communicating the student perspective accurately to the SCS Dean
    \item Communicating progress back to the PhD student body 
    \item Documenting PhD student advocacy in a persistent and transparent archive
\end{itemize}



2) Improve the overall PhD experience for students of all backgrounds and needs. 
This involves:
\begin{itemize}
    \item Identifying student problems that are not being actively worked on
    \item Proposing and launching solutions to ensure all students, not simply the majority, are supported in all aspects of their PhD
    \item Evaluating the results of past efforts to address student problems, both within and outside of our Committee, and using those lessons learned to inform our work
\end{itemize}


We report back to the Dean at least twice a semester. At least once a semester, we report to the student body regarding our projects and focus areas.

To measure progress, we will implement:
\begin{itemize}
    \item Regular reflections on internal efficacy
    \item Semesterly information gathering (e.x. surveys) aimed at PhD students with specific focus related to existing/potential initiatives.
    \item The Committee will also have information gathering that is monitored year-round and available to all constituents (e.g. an anonymous feedback form). 
\end{itemize}


To ensure transparency and the proper representation of student interests, we will publish all activities to a public forum accessible by SCS PhD students.



\subsection {Committee Responsibilities} \label{sec:responsibilities}
The Committee has responsibilities to the SCS Student Body, the Dean's Office, and to Committee members.

\subsubsection{Meetings with Dean} 
The Committee meets with the Dean at least twice a semester, and also on the Dean's request. Working groups or the Committee as a whole may also request additional meetings with the Dean.

At these meetings, Committee members can communicate Working Group progress and discuss ongoing or new initiatives.
Each Working Group is allowed at least one spokesperson to report and discuss progress.% \todo[inline]{This number cannot exceed 15 [Defined by the Dean].}

\subsubsection{General Body Meetings} \label{sec:GBM} 

General Body Meetings will be hosted at minimum monthly.

At these meetings, each Working Group should report progress and goals. In addition, the Working Group representatives can solicit feedback, and Committee members can bring up any other issue they feel would benefit from a high level Committee discussion.

The Spokesperson from every Working Group (including administrative Working Groups) must attend these meetings, and the meetings are open to anyone else who would like to attend.
The General Body Coordination Working Group will send out an agenda to the Committee at least 24 hours before the meeting.
Any Committee member can submit content to be placed on the agenda.

Some General Body Meetings should be set aside for reflection and feedback on the internal workings of the Committee.

\subsubsection{Documentation}

Information in this Committee is public and non-confidential by default. All relevant information should be documented for the student body. This includes aggregated feedback, meeting minutes (with personal information removed), and major communications with the administration. Any documentation pertaining to individual, sensitive matters should not include personally identifiable details without student consent. Sensitive matters pertaining to a large group of students must be aggregated and anonymized prior to documentation.

The committee must maintain a website or central data repository that serves as an entry point for all notes, communications, and documentation. 
We will market this website/repository to the student body at least once a semester.

\subsubsection{Information from the Dean's Office}

Information received from the Dean can be publicly disclosed, unless explicitly told otherwise.

\subsubsection{Feedback from students}

Surveys solicited from the student body will have results reported in aggregate, with sensitive or identifying information removed. 

Other forms of student feedback, such as a student talking about a specific experience in a Working Group meeting, will be documented in an anonymized way.   
If the Committee believes that the notes would benefit from a de-anonymized version of the feedback, they can ask the student for consent in de-anonymizing the information.
The student can request that the documentation be taken down at any point.

\section{Working Groups} \label{sec:workinggroup}

A Working Group is a group of students working towards a specific goal using the resources and framework of the Committee, and aligned with the mission statement of the Committee. The Working Group is the primary organizational unit of the Committee, and being a member of the Committee requires participation in at least one Working Group. Committee members may participate in multiple Working Groups.

\subsection{Creation} \label{sec:creation}
Working Group creation begins with a student proposal for an initiative. The proposal must describe how the initiative aligns with the Committee's mission statement, as well as the name of the Working Group intended to carry out the initiative. Any SCS PhD student (or SCS Masters student with the support of at least two SCS PhD students \footnote{Existing committee members may be solicited for support.}) may create a proposal, not only existing Members of the Committee. A well-publicized template for these proposals must be maintained.

The alignment of the proposal with the Mission Statement will be discussed and decided upon by the Committee within a month after it is received. The criteria for alignment are whether the proposed initiatives falls under one of the two primary goals and responsibilities stated in the Mission Statement, as well as whether the initiative is compatible with the commitments to communication and transparency as outlined in the Mission Statement.

A poll on the proposal will be opened to the general body of committee members before the final General Body Meeting in the allotted decision period (1 month). If the result of the poll receives a simple majority of support, then the proposal is accepted.  Else, the final decision about alignment must be reached by a simple majority vote of attending members of the General Body Meeting. If the initiative is found to not be in alignment with the Mission Statement, a descriptive rebuttal will be provided to the proposer. The proposers can edit the proposal and resubmit for consideration. 

Once an initiative is found to be in alignment with the Mission statement, it becomes an official Working Group. The Student Body Interface Working Group will work with the proposers to finalize the goal and lifetime of the Working Group, and then publicize publicize the newly formed Working Group alongside Working Groups which are recruiting.

\subsection{Goal and Lifetime}
Before a Working Group can be publicized, it must decide on a Goal. The goal must be reasonably described by the name of the Working Group. The goal must be specific in the kinds of actions the Working Group will take, as well as have identify what measurable effects these actions are intended to have. The goal cannot require resources which the Committee cannot gain access to. Goals should be as small and specific as possible to encourage actionable and tangible outcomes.

The lifetime of a Working Group must have a specified end before it can be publicized. The lifetime cannot be longer than the intended engagement of the proposers. It cannot be shorter than the time it would reasonably take to achieve the stated goal. When the lifetime of a Working Group is up, the group can propose another goal and continue functioning under the same name. If the new goal is no longer reasonably described by the name of the group, those involved can go through the Working Group creation process from the beginning.


\subsection{Responsibilities}
Working groups must follow the guidelines in this constitution, but are free to organize any additional structures necessary for their functioning.

Working groups must make consistent progress towards their stated goal, and provide periodic feedback on this progress. Working groups report their progress at General Body Meetings, in communications to the Dean's office, and in the Committee documentation for the student body.

Public documentation for the student body must be updated at least once every month, and should include meeting minutes (if any) and all collected data and relevant communications to non-Committee entities.
 
Working groups must send a representative to attend General Body Meetings and  meetings with the Dean's Office.% \todo{Attendance, how it affects rights for elections}

\subsection{Membership and Roles}
Working group membership is open to any SCS PhD student. 
%\todo{Working group meetings and discussions are open to any SCS student in any program.}
Non-PhD SCS students may join working groups if their respective Advisory Committee does not have a corresponding initiative. Working groups must be at least 2/3 PhD students at all times to ensure the group's efforts are aligned with the interests of the PhD student community.

Working groups must designate a single external Spokesperson whose contact info can be listed with the Working Group information.

Working groups must also designate at least one person who will ensure that the group produces appropriate documentation. This can be the same person as the external Spokesperson, or it can be multiple people. It is the responsibility of this person to maintain or be able to procure an up-to-date list of involved members of the Working Group.

%\section{Membership} \label{sec:membership}
%\subsection{What are the responsibilities of a member?}
%A member of the Committee must join at least one Working Group, and attend General Body Meetings. 
%\subsection{What are the rights of a member?}
%Any member is able to vote in elections for the heads of standing internal Working Groups. 
%\subsection{How do I become a member?}
%Join a Working Group and attend General Body Meetings. 
%\subsection{How do I stop being a member?}
%Leave a Working Group and/or no longer attend General Body Meetings.

% \section{Executive Board (E-board)}\label{sec:eboard}

% \subsection{How do I become an E-board member?}
% Anyone on the Committee can be elected for select internal E-board membership roles. These positions are for internal and administrative tasks that are long-standing unlike the ad-hoc nature of Working Groups. In addition, leaders of Working Groups are automatically E-board members. 
% \subsection{How do I stop being an E-board member?}



\section{Administrative Working Groups}
There are three types of administrative working groups to ensure that the Committee can fulfill its responsibilities. The three working groups are : Dean Interface, Internal Interface, and Student Body Interface.

The process of joining an administrative working group is the same as all other working groups. The points of contact for these administrative working groups are required to attend General Body Meetings and the regular meetings with the Dean.

These working groups have some key differences from non-administrative working groups. They have a Spokesperson who must be elected by the general body of committee members. In addition, these three administrative working groups have an indefinite lifetime, and will continue to operate unless a constitutional amendment changes this. 

\subsection{Elections}\label{sec:elections}

The election will be conducted by an elections committee consisting by default of the members of the Internal Interface working group, excluding any member of that working group who will be up for election. If all members of the Internal Interface working group will be going up for election, it is their responsibility to solicit a special committee from other members of the Committee to conduct the election. This election process will begin a month before the last day of class each semester, and must conclude by the last day of classes.

All SCS PhD students who are involved with at least one working group are eligible to run for the Spokesperson for any of these three administrative working groups. To determine eligibility, the elections committee must send a request to all Working Group Spokespeople for a current list of active participants in their working group. The Spokespeople must provide this list within a week.

Any eligible student can nominate themselves or another eligible student for any of these positions. If a student is nominated by another, the former can accept or reject the nomination. Students going up for election must send a description to the elections committee to be shared with the voting materials. This nomination period lasts one week and begins when the elections committee sends the nomination form to the list of eligible members. 


At the end of the nomination period, the elections committee must create an electronic poll of all the candidates and send the poll and campaign materials to the list of eligible members.

In order to get elected, a candidate must be the Condorcet winner of the election. 

If, at any point, there is no Spokesperson for any administrative working group, the elections committee must hold a special election (with the same procedure as above), starting with a solicitation of active Working Group members within a week of the vacancy.

\subsection{Working Group Accelerator}

The Working Group Accelerator Working Group is responsible for assisting individual working groups and coordinating between working groups. Its overall goal is to maximize the long-term impact of the working groups and act as a communication channel between them.

Members of this working group will regularly meet with or attend the meetings of the non-administrative working groups, identifying overlap in efforts between working groups and sharing knowledge. They will also help the working groups identify and implement solutions to challenges. They will also assist working groups in preparing for updates in General Body Meetings.



%%
% Scheduling general meetings with the Dean and coordinating information shared.
% Making sure we have something to send the Dean (agenda, progress, proposals)
% Meeting deadlines (for meetings with the Dean)
% Maintaining the relationship with the Dean's office; helping members in facilitating that relationship.

% When time sensitive correspondence comes through from the Dean's office, gathering representatives from all the Working Groups and scheduling meetings to respond, in tandem with the General Body Coordinator.

\subsection{Internal Interface} % Internal

The Internal Interface Working Group is responsible for scheduling and running General Body Meetings, keeping track of Committee membership, and handling elections (as outlined in Section \ref{sec:elections}). Its overall goal is to perform tasks that don't belong in any single working group, as well as facilitate communication with the Dean.

General Body Meetings should be organized as outlined in Section \ref{sec:GBM}. This Working Group is also responsible for acting on proposed changes to the committee or feedback given.

They will also handle the scheduling and agenda setting of meetings with the Dean, as well as urgent requests from the Dean's office.

% Scheduling and conducting General Body Meetings
% Tracking attendance at General Body Meetings
% Planning regular reflection and internal feedback
% Making sure that Working Groups are functional and doing conflict resolution/feedback
% Tracking workload and manpower for each group
% Communicating with Working Group leaders
% \todo{Membership management}

\subsection{External Interface} % Constituents

The External Interface Working Group is responsible for maintaining communication with the rest of the SCS PhD student body. In particular, they are responsible for documentation practices, mass communication, onboarding, and processing Working Group proposals. Its overall goal is recruitment, communication with the student body, and onboarding.

Although all Working Groups are responsible for producing documentation, the External Interface Working Group is responsible for maintaining the central repository, sharing and maintaining best practices, and making sure that the documentation is accessible to committee members and the broader student body at all times. 

They advertise the Committee's activities to the student body by way of sharing this central repository, as well as the list of active and recruiting Working Groups. In addition, if a student has a concern they would like to discuss with the committee, the Student Body Interface Working Group is their first point of contact.

Members of the External Interface Working Group are responsible for onboarding new members -- for example, introducing the structure of the committee, adding them to the correct Working Groups(s), and adding them to relevant accounts (i.e. Google Drive, Slack, Github etc).

Finally, proposals for new Working Groups are submitted to this working group, as described in Section \ref{sec:creation}. Members of the External Interface Working Group should assist the proposers in making sure that their proposal is aligned with the Committee's Mission Statement .

% Documentation (website, google drive) - keeping history
% Emailing student body
% Soliciting feedback, managing surveys, assisting Working Groups in soliciting feedback
% \subsubsection{Onboarding}
% Onboarding new members - moving them into Working Groups and adding them to all the relevant accounts (Google Drive, Slack, etc)
% Evaluating proposed Working Groups for alignment with mission
% Publicizing newly created Working Groups
% Making new Working Groups and ensuring people split into groups.

\section{Amendments to this Document}
The Committee has the power and the responsibility to make amendments to this constitution as necessary and proper for fulfilling the Mission Statement of the SCS Dean's PhD Student Advisory Committee. Amendments require a 2/3 majority of attending Members at a General Body Meeting. These meetings and the proposed amendments should be publicized at least a week in advance to all students who are participating in at least one working group. 

\end{document}
